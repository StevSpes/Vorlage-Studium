%Dokumentklasse
\documentclass[a4paper,12pt]{scrreprt}
\usepackage[left= 2cm,top=2cm, right = 4cm, bottom = 2cm]{geometry}
\addtokomafont{disposition}{\rmfamily}

% Dokumentinformationen
\usepackage[
	pdftitle={Auslegung eines Druckbehälters nach Merkblatt AD2000},
	pdfsubject={},
	pdfauthor={Steffen Specker},
	pdfkeywords={},	
	hidelinks
]{hyperref}


% Standard Packages
\usepackage[utf8]{inputenc}
\usepackage[english,ngerman]{babel} %Landessprache Deutsch
\usepackage[T1]{fontenc} % Wörtertrennung etc.

%Graphiken
\usepackage{graphicx, subfig}  		%Zum Einbinden von Bildern 
\usepackage[dvips,final]{graphicx}	%JPG Graphiken einbinden
\graphicspath{{/img}}				%Hier liegen die Bilder


\usepackage{eurosym}
\usepackage{textcomp} 	%Euro Zeichen etc.

%Schriftzeichen 
\usepackage{fancyhdr}	%Kopf und Fußzeilen
\usepackage{color}		%einstellbare Schriftfarben
\usepackage{lmodern} 	%bessere SChrift in der pdf
\usepackage{relsize} 	%Schriftgröße relativ festlegen

%für Index-Ausgabe mit \printindex
\usepackage{makeidx}

%Bessere Unterstreichungen
\usepackage[normalem]{ulem}

% zusätzliche Schriftzeichen der American Mathematical Society
\usepackage{amsfonts}
\usepackage{amsmath}
\usepackage{Times}
\usepackage{mathptmx}

%Einfache Definition der Zeilenabstände und Seitenränder etc.
\usepackage{setspace}
\usepackage{xspace}
\usepackage{geometry}

%Abkürzungsverzeichnis
\usepackage[printonlyused]{acronym}
\usepackage[intoc]{nomencl}

\renewcommand{\nomname}{Abkürzungsverzeichnis und Glossar}
\setlength{\nomlabelwidth}{.25\hsize}
\renewcommand{\nomlabel}[1]{#1 \dotfill}
\setlength{\nomitemsep}{-\parsep}

%nicht einrücken nach Absatz
\setlength{\parindent}{0pt}

%Schönere Tabellen
\usepackage{tabularx} % Tabellen mit automatischen Zeilenumbruch in der Zelle

\usepackage{hyperref}
\usepackage{pdfpages}


% Titelseite beziehungsweise der Inhalt derselbigen:
\title{Titel der Arbeit}
\author{Steffen Specker}
\date{\today}

\onehalfspacing{} %Zeilenabstand auf 1,5 setzen


% ============= Package Einstellungen & Sonstiges ============= 
%Besondere Trennungen
\hyphenation{De-zi-mal-tren-nung}


\newcommand{\titel}{Auslegung eines Dampfdruckbehälters nach Merkblatt AD2000}
\newcommand{\untertitel}{}
\newcommand{\untertitelDeckblatt}{}
\newcommand{\art}{Praxistransferbericht}
\newcommand{\fachgebiet}{zur Erlangung des akademischen Grades\\ Bachelor of Engineering (B.\,Eng.) im Studienfach\xspace}
\newcommand{\autor}{Steffen Specker}
\newcommand{\keywords}{Praxistransferbericht, Steffen Specker}
\newcommand{\studienbereich}{Maschinenbau\xspace}
\newcommand{\matrikelnr}{172897}
\newcommand{\erstgutachter}{Herr Kray}
\newcommand{\zweitgutachter}{Dipl.-Ing. (FH) Herbert Beispiel}
\newcommand{\jahr}{2022}
\newcommand{\hochschule}{Private Hochschule für Wirtschaft und Technik}
\newcommand{\ort}{Diepholz}
\newcommand{\logo}{PHWT-Logo.pdf}
\newcommand{\creator}{}

% ============= Dokumentbeginn =============

\begin{document}
\ofoot{}
\thispagestyle{plain}

%Nummerierungsebenen im Inhaltsverzeichnis, Deckblatt ohne Kopf- und Fußzeile sowie Seitenzahl
\setcounter{secnumdepth}{3}
\setcounter{tocdepth}{2}

\begin{titlepage}


\begin{center}
\begin{tabular}{p{\textwidth}}

\\

\begin{center}
\large {Private Fachhochschule für Wirtschaft und Technik \\
Bachelor of Engineering\\
Ausbildungsbetrieb: DIL Engineering GmbH}
\end{center}

\\


\begin{center}
\includegraphics[scale=1]{img/PHWT-Logo.jpg}
\end{center}

\\
\\

\begin{center}
\large{Praxistransferbericht zum Thema:}
\end{center}

\\

\begin{center}
\Large{Auslegung eines Druckbehälters nach Merkblatt AD2000}
\end{center}


\begin{center}
Beschreibung 1\\
Beschreibung 2
\end{center}

\\

\begin{center}
vorgelegt von: 
\end{center}

\begin{center}
    \begin{tabular}{lll}
        \textbf{Steffen Specker:} & & Matrikelnr. 172897\\
        \end{tabular} 
\end{center}

\\

\\

\\

\begin{center}
\begin{tabular}{lll}
\textbf{Prüfer:} & & Herr Kray\\
\textbf{Abgabedatum:} & & 31.07.2022\\
\end{tabular}
\end{center}

\end{tabular}
\end{center}

\end{titlepage}
\ofoot{\pagemark\\[4ex]}

%Seitennummerierung: Vor dem Hauptteil in großen Römischen Buchstaben 

\pagenumbering{Roman}
\phantomsection{} % Sorgt für korrekte Aufnahme des Inhaltsverzeichnisses in das Inhaltsverzeichnis
\addcontentsline{toc}{chapter}{Inhaltsverzeichnis}
\tableofcontents{}
\renewcommand{\thepage}{\Roman{page}}


% pagestyle für gesamtes Dokument aktivieren
%\pagestyle{fancy}

%Verzeichnis aller Bilder
\listoffigures

%Verzeichnis aller Tabellen
\listoftables 

\addchap{Abkürzungsverzeichnis}\label{Abkürzungsverzeichnis} 
\begin{acronym}[GHD] 
   \acro{GHD}{Gewerbe, Handel und Dienstleistung} 
   \acro{KDE}{K Desktop Environment}
   \acro{SQL}{Structured Query Language}
   \acro{Bash}{Bourne-again shell}
   \acro{JDK}{Java Development Kit}
   \acro{VM}{Virtuelle Maschine}
   \acro{I2C}[I²C]{Inter-Integrated Circuit}
\end{acronym} 


\setcounter{page}{0}
\pagenumbering{arabic}
\renewcommand{\thepage}{\arabic{page}}


\chapter{Einleitung}\label{sec:einleitung}
Weit hinten, hinter den Wortbergen. Lalala.

Dies ist ein Test. Beispiel Abkürzung:


\chapter{Grundlagen}\label{sec:grundlagen}


\section{Beispielkapitel}\label{sec:beispiel}
\begin{figure}[htb]
  \centering  
  \includegraphics[scale=0.5]{img/starwars.jpg}
  \caption{Star Wars Logo}\label{fig:starwars}
\end{figure}
\\

Test Test.\\
Der \ac{KDE} ist ein großes Organ der EU.\


\section{Beispielunterkapitel}\label{subsec:beispiel}


%Neues Kapiten anfangen mit \chapter*{Titel}
\chapter{Stand der Technik}\label{cha:stand_der_technik}

\section{Klassifizierung von Beispielen}\label{sec:klassifizierung}
\section{Themenbezogene Veröffentlichungen}\label{sec:themenbezogene_veroeffentlichungen}

\chapter{Entwicklung der Sensorik}\label{chap:entwicklung}
\section{Grobkonzept der Sensorik}\label{sec:grobkonzept}

\chapter{Ergebnisse}\label{sec:ergebnisse}

\chapter{Diskussion}\label{sec:diskussion}
\section{Zusammenfassende Bewertung}\label{sec:überschrift}

\section{Ausblick}\label{sec:ausblick}

\include{08_Anhang.tex}


%Literaturverzeichnis
%\bibliographystyle{unsrtdin}
%\bibliography{Literatur}


%Erklärung ohne Nummerierung im Inhaltsverzeichnis
\addsec{Eidesstattliche Erklärung}\label{erklaerung}

Hiermit versichere ich, die vorliegende Abschlussarbeit selbstständig und nur unter Verwendung der von mir angegebenen Quellen und Hilfsmittel verfasst zu haben. Sowohl inhaltlich als auch wörtlich entnommene Inhalte wurden als solche kenntlich gemacht. Die Arbeit hat in dieser oder vergleichbarer Form noch keinem anderem Prüfungsgremium vorgelegen. \\
\\[1.5cm]
Datum:\hrulefill\enspace{} Unterschrift: \hrulefill{}
\\[3.5cm]
\ort, den \today

\rule[-0.2cm]{5cm}{0.5pt}

\textsc{\autor} 

\end{document}
