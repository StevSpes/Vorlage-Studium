\documentclass[
    12pt, % Schriftgröße
    ngerman, % für Umlaute, Silbentrennung etc.
    a4paper, % Papierformat
    oneside, % einseitiges Dokument
    headings=big, % Größe der Überschriften verkleinern
    nolistof=totoc, % Verzeichnisse im Inhaltsverzeichnis aufführen
    nobibliography=totoc, % Literaturverzeichnis im Inhaltsverzeichnis aufführen
    index=totoc, % Index im Inhaltsverzeichnis aufführen
    captions=tableheading, % Beschriftung von Tabellen unterhalb ausgeben
    final % Status des Dokuments (final/draft)
    %toc=chapterentrywithdots %Inhaltsverzeichnis mit Punkten
    sectionentrydots=true,
    toc = bibliography
]{scrreprt}
%scrartcl -> Basis der Vorlage = Koma Script

%Allgemeine Pakete
\usepackage[utf8]{inputenc} 
\usepackage[ngerman]{babel}                         %Landessprache Deutsch
\usepackage[T1]{fontenc}                            %Schriftformatierung

% Graphik- und Tabellenpakete
\usepackage[pdftex]{graphicx}                      %Implementierung 
\usepackage{pdfpages}
%Grafiken
\graphicspath{{img/} }                              % Bildordner
\usepackage{subfig}                                 % Mehrere Graphiken nebeneinander
\usepackage{tabularx}                               %Tabellenpaket
\usepackage{tabulary}                               %Tabellenbreite einstellen

%Tabellen für Formeln und Gleichungen
\usepackage{amsmath}

%Hyperlinks (erst am Ende aktivieren)
%\usepackage[colorlinks,pdfpagelabels,pdfstartview = FitH,bookmarksopen = true,bookmarksnumbered = true,linkcolor = black,plainpages = true,hypertexnames = false,citecolor = black] {hyperref} %Erstellen von Hyperlinks
%\usepackage[figure]{hypcap}                        %Hyperlinks für Abbildungen


%Schriftdarstellung und Textoptimierung 
\usepackage{mathptmx}               % Times New Roman
\usepackage{microtype}              % Blocksatzbildung
\hyphenation{De-zi-mal-tren-nung}   % Deutsche Zeilenbrüche
\usepackage{color}                  % Möglichkeit zum Verwenden von Farben
\usepackage{lmodern}                % bessere Fonts
\usepackage{relsize}                % Schriftgröße relativ festlegen
\usepackage{blindtext}              %Packet Blindtext zum Einfügen von sinnlosen Text


% Pakete für Geometrie und Abstände
\usepackage{geometry} %zunächst auf Standardeinstellungen belassen...
\usepackage{setspace} 



%Verzeichnispakete 
\usepackage[automark]{scrlayer-scrpage}             % Kopf und Fußzeile
\usepackage[printonlyused,withpage]{acronym}        % Abkürzungsverzeichnis
\usepackage[final]{listofsymbols}                   % Symbolverzeichnis
\usepackage{eurosym}
\usepackage{textcomp}

\providecaptionname{ngerman}{\listoflofentryname}{Abbildung}
\providecaptionname{ngerman}{\listoflotentryname}{Tabelle}
\BeforeStartingTOC[lof]{\def\autodot{:}}
\BeforeStartingTOC[lot]{\def\autodot{:}}

\usepackage[citestyle=numeric-verb,backend=biber,style=numeric]{biblatex}
\usepackage[autostyle,german=guillemets,german=quotes]{csquotes}
\addbibresource{literaturverzeichnis.bib}

%Schriftarten von Section, Chapter, Text etc. einstellen, Format Inhaltsverzeichnis
\renewcommand{\rmdefault}{ptm}
\renewcommand{\sfdefault}{phv}
\renewcommand\familydefault{\rmdefault}
\addtokomafont{chapter}{\rmfamily\mdseries\Large \hspace*{-0.2cm}\vspace{-10pt}}
\addtokomafont{section}{\normalfont\large\rmfamily\mdseries\hspace*{-0.2cm}\vspace{-10pt}}
\addtokomafont{chapterentry}{\rmfamily\mdseries\large \hspace{0pt}\vspace{0pt}}
\addtokomafont{partentry}{\vspace{-30pt}\rmfamily\mdseries\large \hspace{0pt}}

%Seitenabstände und Abstände zwischend den Kapiteln
\RedeclareSectionCommand[beforeskip=0pt,afterskip=\baselineskip,afterindent=false]{chapter}
\RedeclareSectionCommand[beforeskip=0pt,afterindent=false]{section}

%Vertikalen Abstand zwischen Verzeichnissen und eigentlichem Text einfügen
\DeclareTOCStyleEntry[
  onstarthigherlevel=\addvspace{4em}
]{part}{chapter}

\AtBeginDocument{%
  \newcaptionname{ngerman}\xequationname{Formel}%
  \newcaptionname{ngerman}\listxequationname{Formelverzeichnis}%
}


% Kopf- und Fußzeile der ersten Seiten (Paket scrlayer-scrpage)
\pagestyle{scrheadings}
\clearmainofpairofpagestyles{}  % alle Felder leeren
%\ihead{}\chead{}\chead{}
\cfoot{\pagemark}               % Seitenzahl oben rechts


\begin{document}
\thispagestyle{plain}

%Nummerierungsebenen im Inhaltsverzeichnis, Deckblatt ohne Kopf- und Fußzeile sowie Seitenzahl
\setcounter{secnumdepth}{3}
\setcounter{tocdepth}{2}

\begin{titlepage}


\begin{center}
\begin{tabular}{p{\textwidth}}

\\

\begin{center}
\large {Private Fachhochschule für Wirtschaft und Technik \\
Bachelor of Engineering\\
Ausbildungsbetrieb: DIL Engineering GmbH}
\end{center}

\\


\begin{center}
\includegraphics[scale=1]{img/PHWT-Logo.jpg}
\end{center}

\\
\\

\begin{center}
\large{Praxistransferbericht zum Thema:}
\end{center}

\\

\begin{center}
\Large{Auslegung eines Druckbehälters nach Merkblatt AD2000}
\end{center}


\begin{center}
Beschreibung 1\\
Beschreibung 2
\end{center}

\\

\begin{center}
vorgelegt von: 
\end{center}

\begin{center}
    \begin{tabular}{lll}
        \textbf{Steffen Specker:} & & Matrikelnr. 172897\\
        \end{tabular} 
\end{center}

\\

\\

\\

\begin{center}
\begin{tabular}{lll}
\textbf{Prüfer:} & & Herr Kray\\
\textbf{Abgabedatum:} & & 31.07.2022\\
\end{tabular}
\end{center}

\end{tabular}
\end{center}

\end{titlepage}

%Absätze einstellen
\newgeometry{left=4cm, right=2cm, top=2cm}
\setlength{\parindent}{0cm}                     % Einrücken des Textes bei neuem Absatz
\setlength{\parskip}{0.5cm}                     % Vertikaler Abstand der Absätze
\doublespacing{}                                % Zeilenabstand

% Seitenränder für das Textdokument einstellen
\setlength{\headheight}{1.5cm}                  % Höhe Kopfzeile
\setlength{\voffset}{2.5cm}                     % Abstand Oberer Rand - Textfeld
\setlength{\topmargin}{-4.5cm}                  % Abstand Oberer Rand - Text oben
\setlength{\headsep}{0.5cm}                     % Abstand Kopfzeile - Text oben 
\setlength{\footskip}{2cm}                      % Abstand Fußzeile - Text unten
\setlength{\footheight}{1.5cm}                  % Höhe Fußzeile
%Abstand Gleitumgebungen zum Text
\setlength{\intextsep}{12pt}
\setlength{\textfloatsep}{12pt}


\thispagestyle{empty}
%\addcontentsline{toc}{section*}{Eidesstattliche Erklärung}

\section*{Eidesstattliche Erklärung}
Hiermit versichere ich, Steffen Specker, dass ich die vorlegende Ausarbeitung eigenständig und nur unter Zuhilfenahme der im Literaturverzeichnis genannten Werke gefertigt habe. Jede Textpassage, die wörtlich oder dem Sinn nach auf fremdes Gedankengut zurückgreift, wurde als solche kenntlich gemacht.\par

Die Arbeit wurde bisher in gleicher oder ähnlicher Form keiner anderen Prüfungsbehörde vorgelegt und auch noch nicht veröffentlicht.

\bigskip
\bigskip

Eggermühlen, den \today

\bigskip

\rule[-0.2cm]{5cm}{0.5pt}\\
\textsc{Steffen Specker} 


% Formatierung für mehrere Unterschriften nebeneinander
%
%\begin{tabular}{l l}
%    \centering
%    \rule[-0.2cm]{5cm}{0.5pt} & \hspace{3cm} \rule[-0.2cm]{5cm}{0.5pt}  \\
%        \textsc{Steffen Specker} & \hspace{3cm} \textsc{Leon Witteveen}
%\end{tabular}

\pagenumbering{Roman}
%\hypersetup{pageanchor=true} %Hyperlinks erstellen


\tableofcontents
%\addtocontents{toc}{~\hfill\textbf{Seite}\par}

\cleardoublepage{}%wichtig, sonst falsche Seitennr.
\addcontentsline{toc}{part}{Inhaltsverzeichnis}
%\setcounter{secnumdepth}{2}

\newpage

\KOMAoptions{listof=entryprefix}
\addtocontents{lot}{\protect\addcontentsline{toc}{part}{\listtablename}}
\listoftables

%\listoftables % Tabellenverzeichnis
\vspace{2cm}
\addtocontents{lof}{\protect\addcontentsline{toc}{part}{\listfigurename}}
\listoffigures  %Abbildungsverzeichnis


\vspace{2cm}
\cleardoublepage{}
\chapter*{Abkürzungsverzeichnis} 
\addcontentsline{toc}{part}{Abkürzungsverzeichnis} 

\begin{acronym}[Bash]
%zu deklarieren mit: 
%\acro{Abkürzung Eintrag}{Lang ausformuliert}
% plural mit: \acroplural{dr}[Dres.]{Doktoren}

\acro{KDE}{K Desktop Environment}
\acro{dr}[Dr.]{Doktor}
\acroplural{dr}[Dres.]{Doktoren}
\acro{SQL}{Structured Query Language}
\acro{Bash}{Bourne-again shell}
\acro{JDK}{Java Development Kit}
\acro{VM}{Virtuelle Maschine}
\acro{I2C}[I²C]{Inter-Integrated Circuit}
\end{acronym}


\vspace{2cm}

% Symbole für das Symbolverzeichnis deklarieren:
% nur mit \symE (Abkürzung) verwendete Symbole werden gelistet. 
\opensymdef\newsym[Energie in KJ]{symE}{E}
\newsym[Masse in kg]{symm}{m}
\newsym[Lichtgeschwindigkeit in Meter pro Sekunde]{symc}{c}


\closesymdef{}


\renewcommand{\symheadingname}{Symbolverzeichnis}
\cleardoublepage{}
\addcontentsline{toc}{part}{Symbolverzeichnis}
\listofsymbols{}


\vspace{2cm}

\cleardoublepage{}
\chapter*{Formelverzeichnis} 
\addcontentsline{toc}{part}{Formelverzeichnis} 
\renewcommand{\listtablename}{Formel}


%Ab hier beginnt das eigentliche Dokument

\newpage


\chapter{Einführung in das Thema}
\pagenumbering{arabic}



This will be an empty chapter
Vor Jahren waren \ac{KDE} die größten Vermittler der Welt. \\
Des lässt sich auch an xx feststellen. Der Mehrwert Bla Bla \blindtext\par


\blindtext{}

\section{Unterkapitel}
\blindtext{}
\section{Noch eins}
\section{Ein weiteres}

\newpage

\chapter{Grundlagen der Sterilisation}
\blindtext[1]

\par

\begin{figure}[htb]
    \centering  
    \includegraphics[keepaspectratio,width=\textwidth,height=\textheight]{render.png}
    \caption{3D Darstellung des Dampfdruckbehälters}\label{fig:render}
\end{figure}

\par

Dies ist auch in Abbildung~\ref{fig:render} sehr gut zu erkennen (vgl. Abb.\ref{fig:render}).

\blindtext[2] \par


\section{Wahl der Dichtmittel}
\blindtext[2] \par

\section{Einflussgrößen auf die Auslegung nach AD2000}
\blindtext[1]

\begin{table}[htb]
\centering
\begin{tabular}{|c|cc|} 
 \hline
 Einheit & Benennung & Kategorie \\ %[0.5ex] 
 \hline
 m & 6 & 8783 \\ 
 s & 7 & 78 \\
 V & 545 & 778\\
 A & 545 & 18744\\
 \hline
\end{tabular}
\caption{Eine sehr vielaussagende Tabelle}\label{vielaussagend}
\end{table}

\blindtext[1]
\newpage


\chapter{Kapitel mit einige Formeln}
\blindtext{}

\begin{equation}
   a=b
    \label{eq:Eq3}
\end{equation}

\begin{align}
\intertext{ Zum einen gilt: } 
a+b&=c
\intertext{zum anderen jedoch auch } 
b=c
\end{align}

Also was ist folglich c?

\blindtext{}

\begin{equation}\label{eq:Eq7}
a=b
\end{equation}



\begin{equation*}\label{eq:Eq2}
   b=c
\end{equation*}

\begin{equation}
    \label{eq:Eq10}
    Y=Kd \ast \left(Xd + \frac{1}{Tn} + \int Xd\ dt + d\ \frac{Xd}{dt}\right)
\end{equation}


\newpage

\chapter{Kapitel mit Symbolen}

Hier stehen gleich einige Symbole, die im Symbolverzeichnis aufgelistet werden sollen.~\cite[S.~49]{Wittel.2021b}
\[\symE=\symm \symc^2\]~\cite{VerbandderTUVe.V..2020}

where \symE~is the energy \ldots


\newpage
\chapter{Fazit und sicherheitstechnische Beurteilung der Facharbeit}
\blindtext{}
\par
\blindtext{}

\newpage
\pagenumbering{Roman}
\setcounter{page}{8}    %manuell römische Seitenzahl einstellen! 

\begin{appendix}
\clearpage{}

%Erste Seite eines pdf-Anhangs mit Überschrift:
\includepdf[scale=0.9,pages=1,pagecommand={\chapter{Ein PDF Anhang} }, offset=0 0.5cm]{test3}
%Weitere Seiten ohne zusätzliche Überschrift:
\includepdf[scale=0.9,pages=2-3,pagecommand={}, offset=0 0.5cm]{test3} %Pagecommand, dafür dass die Seitenzahl nicht verschwindet



%Graphiken einfügen
\chapter{Ein Tabellenanhang}
\begin{figure}[ht]
	\centering
  \includegraphics[width=\textwidth] {Anhang/tabelle1.png}
	%\caption{um 30 Grad gedreht}
	%\label{fig1}
\end{figure}

\end{appendix}


\newpage
\printbibliography[title=Literaturverzeichnis]

\end{document}

