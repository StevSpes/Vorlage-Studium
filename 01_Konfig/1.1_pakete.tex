

%Dies sind die im Main verwendeten Pakete. Genauere Optionen und Einstellungen finden sich unter der cTan des jeweiligen Pakets. (https://ctan.org/)


%Allgemeine Pakete
\usepackage[utf8]{inputenc} 
\usepackage[ngerman]{babel}                         %Landessprache Deutsch
\usepackage[T1]{fontenc}                            %Schriftformatierung

% Graphik- und Tabellenpakete
\usepackage[pdftex]{graphicx}                      %Implementierung 
\usepackage{pdfpages}
\graphicspath{{img/} }                              % Ordner für die verwendeten Bilder
\usepackage{subfig}                                 % Paket für mehrere Graphiken nebeneinander
%\usepackage{tabularx}                               % Tabellenpaket 1
%\usepackage{tabulary}                               % Tabellenpaket 2
\usepackage{array}


%Tabellen für Formeln und Gleichungen
\usepackage{amsmath}


%Schriftdarstellung und Textoptimierung 
\usepackage{mathptmx}               % Schrifteinstellung für Times New Roman
\usepackage{microtype}              % Blocksatzbildung
\hyphenation{De-zi-mal-tren-nung}   % Verwendung deutscher Zeilenbrüche
\usepackage{color}                  % Möglichkeit zum Verwenden von Farben
\usepackage{lmodern}                % bessere Fonts
\usepackage{relsize}                % Schriftgröße relativ festlegen
\usepackage{blindtext}              %Packet Blindtext zum Einfügen von sinnlosen Text

% Pakete für Blattgeometrie und Seitenabstände - genaue Einstellungen unter einstellungen.tex
\usepackage{geometry}               %zunächst auf Standardeinstellungen belassen...
\usepackage{setspace}               %Paket für vertikale und horizontale Abstände  


%Verzeichnispakete 
\usepackage[automark]{scrlayer-scrpage}             % Paket für Kopf und Fußzeile
\usepackage[printonlyused,withpage]{acronym}        % Abkürzungsverzeichnis
\usepackage{chngcntr}                               % Abbildungsverzeichnis
\usepackage[final]{listofsymbols}                   % Symbolverzeichnis
\usepackage{eurosym}                                % Europäsche Symbole
\usepackage{textcomp}                               % Weitere Symbole



%Literaturverzeichnispakete und Zitierstileinstellung 
\usepackage[citestyle=numeric-verb,backend=biber,style=numeric]{biblatex}
\usepackage[autostyle,german=guillemets,german=quotes]{csquotes}
\addbibresource{literaturverzeichnis.bib}


% Pakete für Hyperlinks (erst ganz am Ende aktivieren!!)
%\usepackage[colorlinks,pdfpagelabels,pdfstartview = FitH,bookmarksopen = true,bookmarksnumbered = true,linkcolor = black,plainpages = true,hypertexnames = false,citecolor = black] {hyperref} %Erstellen von Hyperlinks
%\usepackage[figure]{hypcap}                        %Hyperlinks für Abbildungen
