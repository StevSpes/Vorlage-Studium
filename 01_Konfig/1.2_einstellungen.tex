%
%
% Dies sind die vor Grundeinstellungen für das Dokument. Hier werden Verzeichnisse umbenannt, Seitenabstände, Blattgeometrien, Kopf/Fußzeilen eingerichtet. 



%Schriftarten von Überschriften und Inhaltsverzeichniseinträgen einstellen:
\renewcommand{\rmdefault}{ptm}
\renewcommand{\sfdefault}{phv}
\renewcommand\familydefault{\rmdefault}
\addtokomafont{chapter}{\rmfamily\mdseries\Large \hspace*{-0.2cm}\vspace{-10pt}}
\addtokomafont{section}{\normalfont\large\rmfamily\mdseries\hspace*{-0.2cm}\vspace{-10pt}}
\addtokomafont{subsection}{\normalfont\large\rmfamily\mdseries\hspace*{-0.2cm}\vspace{-10pt}}
\addtokomafont{chapterentry}{\rmfamily\mdseries\large \hspace{0pt}\vspace{-0pt}}
\addtokomafont{part}{\rmfamily\mdseries\Large \hspace*{-0.2cm}\vspace{-10pt}}
\addtokomafont{partentry}{\vspace{-30pt}\rmfamily\mdseries\large \hspace{0pt}}

%Seitenabstände und Abstände zwischend den Kapiteln einstellen
\RedeclareSectionCommand[beforeskip=0pt,afterskip=\baselineskip,afterindent=false]{chapter}
\RedeclareSectionCommand[beforeskip=0pt,afterindent=false]{section}
\RedeclareSectionCommand[beforeskip=-10pt,afterskip=\baselineskip,afterindent=false]{subsection}

%Vertikalen Abstand zwischen Verzeichnissen und eigentlichem Text einfügen
\DeclareTOCStyleEntry[
  onstarthigherlevel=\addvspace{4em}
]{part}{chapter}
\DeclareTOCStyleEntry[
 onstarthigherlevel=\addvspace{3em}
]{section}{chapter}



%Umbenennen und Einstellen für das Tabellen und Abbildungsverzeichnsi
\KOMAoptions{listof=entryprefix}                                % Einstellung: Formulierung vor Verzeichniseintrag
\providecaptionname{ngerman}{\listoflofentryname}{Abbildung}    % "Abbildung" vor vor dem jeweiligen Vz. Eintrag
\providecaptionname{ngerman}{\listoflotentryname}{Tabelle}      % "Tabelle" vor vor dem jeweiligen Vz. Eintrag
\BeforeStartingTOC[lof]{\def\autodot{:}}                        % Doppelpunktsetzung Abbildungsverzeichnis
\BeforeStartingTOC[lot]{\def\autodot{:}}                        % Doppelpunktsetzung Tabellenverzeichnis
\counterwithout{figure}{chapter}                                % Verzeichniseinträge ohne Kapitelbezug
\counterwithout{table}{chapter}                                 

% Symbolverzeichniseinstellungen
\renewcommand{\symheadingname}{Symbolverzeichnis}               % Umbenennen
\setlength{\symwidth}{4cm}                                      % Abstand zwischen Symbol und Erklärung einstellen 


% Kopf- und Fußzeile der ersten Seiten (Paket scrlayer-scrpage)
\pagestyle{scrheadings}
\clearmainofpairofpagestyles{}  % alle Felder leeren
%\ihead{}\chead{}\chead{}
\cfoot{\pagemark}               % Seitenzahl unten mittig


%Absätze einstellen
\setlength{\parindent}{0cm}                     % Einrücken des Textes bei neuem Absatz
\setlength{\parskip}{0.5cm}                     % Vertikaler Abstand der Absätze
\doublespacing{}                                % Allgemeiner Zeilenabstand
%\setlength\abovecaptionskip{2cm}               % Abstand Tabellenunterschrift
%\mathsurround=3cm 

% Seitenränder für das Textdokument einstellen
\newgeometry{left=4cm, right=2cm, top=2cm}
\setlength{\headheight}{1.5cm}                  % Höhe Kopfzeile
\setlength{\voffset}{2.5cm}                     % Abstand Oberer Rand - Textfeld
\setlength{\topmargin}{-4.5cm}                  % Abstand Oberer Rand - Text oben
\setlength{\headsep}{0.5cm}                     % Abstand Kopfzeile - Text oben 
\setlength{\footskip}{2cm}                      % Abstand Fußzeile - Text unten
\setlength{\footheight}{1.5cm}                  % Höhe Fußzeile

%Abstand Gleitumgebungen zum Text
\setlength{\intextsep}{12pt}
\setlength{\textfloatsep}{12pt}


%\overfullrule=20pt

%Abstände Tabelle:
%\usepackage{caption}
%\captionsetup[table]{position=below,skip=2cm}
%\setlength{\abovecaptionskip}{2cm}             %Abstand Tabelle - nachfolgenden Text




