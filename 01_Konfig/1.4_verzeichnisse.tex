%
%
% Verzeichniseinstellungen
% Schriftart, Abstände etc. werden in einstellungen.tex formatiert. 
% 
%
%\hypersetup{pageanchor=true} %Hyperlinks erstellen

% Seitenränder für das Textdokument einstellen
\newgeometry{left=4cm, right=2cm, top=2cm}
\setlength{\topskip}{\ht\strutbox}              % Beseitigt Fehlermeldung
\setlength{\headheight}{1.5cm}                  % Höhe Kopfzeile
\setlength{\voffset}{2.5cm}                     % Abstand Oberer Rand - Textfeld
\setlength{\topmargin}{-4.5cm}                  % Abstand Oberer Rand - Text oben
\setlength{\headsep}{0.5cm}                     % Abstand Kopfzeile - Text oben 
%\setlength{\footskip}{0cm}                    % Abstand Fußzeile - Text unten
\setlength{\footheight}{25pt}                  % Abstand Fußzeile - unten

%Abstand Gleitumgebungen zum Text
\setlength{\intextsep}{12pt}
\setlength{\textfloatsep}{12pt}
\raggedbottom{}                                 % Seite nicht vom unteren Rand ausrichten

%\overfullrule=20pt




%Inhaltsverzeichnis einfügen
\tableofcontents
\addcontentsline{toc}{part}{Inhaltsverzeichnis} %Eintrag im LitVz. mit eig. Einst. "part"
\cleardoublepage{}                              %wichtig, sonst falsche Seitennr.
%\setcounter{secnumdepth}{2}                    %Tiefe der aufgelisteten Unterkapitel


%Tabellenverzeichnis
\addtocontents{lot}{\protect\addcontentsline{toc}{part}{\listtablename}}
\listoftables
\cleardoublepage{}
% \vspace{2cm} und weitere Einstellung notwendig, falls Verzeichnisse auf einer Seite....

%Abbildungsverzeichnis
\addtocontents{lof}{\protect\addcontentsline{toc}{part}{\listfigurename}}
\listoffigures
\cleardoublepage{}

%Abkürzungsverzeichnis
\cleardoublepage{}
\chapter*{Abkürzungsverzeichnis} 
\addcontentsline{toc}{part}{Abkürzungsverzeichnis} 


\begin{acronym}[Bash] %Bash ist die längste Abkürzung, fürs Format

%zu deklarieren mit: 
%\acro{Abkürzung Eintrag}{Lang ausformuliert}
% plural mit: \acroplural{dr}[Dres.]{Doktoren}

\acro{KDE}{K Desktop Environment}
\acro{dr}[Dr.]{Doktor}
\acroplural{dr}[Dres.]{Doktoren}
\acro{SQL}{Structured Query Language}
\acro{Bash}{Bourne-again shell}
\acro{JDK}{Java Development Kit}
\acro{VM}{Virtuelle Maschine}
\acro{I2C}[I²C]{Inter-Integrated Circuit}

\end{acronym}


% Symbole für das Symbolverzeichnis deklarieren:
% nur mit \symE (Abkürzung) verwendete Symbole werden gelistet. 
%Symbole müssen manuell sortiert werden


\opensymdef

\newsym[Energie in KJ]{symE}{E}
\newsym[Masse in kg]{symm}{m}
\newsym[Lichtgeschwindigkeit in Meter pro Sekunde]{symc}{c}

\closesymdef{}



%Symbolverzeichnis
\cleardoublepage{}
\addcontentsline{toc}{part}{Symbolverzeichnis}
\listofsymbols{}


%Formelverzeichnis
\cleardoublepage{}
%\chapter*{Formelverzeichnis} 

\listof{formel}{Formelverzeichnis}
\addcontentsline{toc}{part}{Formelverzeichnis} 

\newpage
\onehalfspacing{}