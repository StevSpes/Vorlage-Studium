% Dies ist ein Beispielkapitel mit verschiedenen Anwendungen.




\chapter{Einführung in das Thema}           % Kapitelüberschrift, Ebene 1
\pagenumbering{arabic}                      % nur beim ersten Kapitel auswählen

%Abkürzungen verwenden: \ac{Name_Abkürzung} anstelle der Abkürzung einsetzen


This will be an empty chapter
Vor Jahren waren \ac{KDE} die größten Vermittler der Welt.\par % einzeiliger Zeilenumbruch (vgl. Str. + Enter)

Das lässt sich auch an xx feststellen.\blindtext \par % Zweizeiliger Zeilenumbruch ( Vgl. Enter)

\blindtext{}

\section{Unterkapitel}
\blindtext{}



\chapter{Grundlagen der Sterilisation}
\blindtext[1] \par
%Bild einfügen, Bildbezug im Text über ~\ref{fig:render} 

\begin{figure}[htb]
    \centering  
    \includegraphics[keepaspectratio,width=\textwidth]{render.png}
    \caption{3D Darstellung des Dampfdruckbehälters}\label{fig:render}
\end{figure}

\par

Dies ist auch in Abbildung~\ref{fig:render} sehr gut zu erkennen (vgl. Abb.\ref{fig:render}).


\blindtext[2] \par


\section{Wahl der Dichtmittel}
\blindtext[2] \par

\section{Einflussgrößen auf die Auslegung nach AD2000}
\blindtext[1]


%Tabelle einfügen: 

\begin{table}[htb]
\centering
\begin{tabular}{|c|cc|} 
 \hline
 Einheit & Benennung & Kategorie \\ %[0.5ex] 
 \hline
 m & 6 & 8783 \\ 
 s & 7 & 78 \\
 V & 545 & 778\\
 A & 545 & 18744\\
 \hline
\end{tabular}
\caption{Eine sehr vielaussagende Tabelle}\label{vielaussagend}
\end{table}


Wie in Tabelle \ref{vielaussagend} zu erkennen, handelt es sich um eine sehr unwichtige Information, die hier gelistet ist.\blindtext[1]


\newpage
%\chapter{Kapitel mit einigen Formeln}
\blindtext{}


%Aufschreiben von Formeln:
%1. Möglichkeit: Equatation - Umfeld
% für eine Darstellung im Formelverzeichnis
% Caption: Beschriftung unter der Formel und im Verzeichnis
% Label: Referenz für Textverweise

\begin{formel}
\begin{equation}
   a=b\label{eq:Eq3}          %Label Referenz für Textverweise
\end{equation}
\caption{Grundformel 1}
\end{formel}

\begin{formel}
\begin{equation}
1+1=3\label{formel2}
\end{equation}
\caption{Schwerkraftsberechnung}
%\label{Berechnung der Schwerkraft}
\end{formel}


Mit diesen beiden Formeln ergeben sich verschiedene Möglichkeiten der Umformung, die für den weiteren Verlauf genutzt werden können:

%2. Möglichkeit: align Umgebung
%-> Vorteil: Intertext, Formatierung etc.

\begin{align}
a+b&=c
\intertext{Des Weiteren gilt:} 
b=c
\end{align}

Also was ist folglich c? Nach dem Tüv \cite[vgl.][S.~133]{VerbandderTUVe.V..2020} gilt:

\begin{formel}
\begin{equation}
    \label{eq:Eq10}
    Y=Kd \ast \left(Xd + \frac{1}{Tn} + \int Xd\ dt + d\ \frac{Xd}{dt}\right)
\end{equation}
\caption{Komplizierte Formel}
\end{formel}




\newpage
\chapter{Kapitel mit Symbolen und Zitaten}

Hier stehen gleich einige Symbole, die im Symbolverzeichnis aufgelistet werden sollen. Zu finden sind die Quellen auch im Literaturverzeichnis unter~\cite[S.~49]{Wittel.2021b}
\[\symE=\symm \symc^2\]

where \symE~is the energy \ldots
Die Formel entstammt der TüV eV \cite[S.~133]{VerbandderTUVe.V..2020}



\newpage
\chapter{Fazit und sicherheitstechnische Beurteilung der Facharbeit}
\blindtext{}
\par
\blindtext{}
